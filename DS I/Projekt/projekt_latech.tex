% --- PDFLATEX PREAMBLE BLOCK ---
\documentclass[11pt, a4paper]{article}

% Geometrie stránky
\usepackage[a4paper, top=2.5cm, bottom=2.5cm, left=2.5cm, right=2.5cm]{geometry}

% Kódování a fonty pro pdfLaTeX
\usepackage[utf8]{inputenc} % Vstupní kódování
\usepackage[T1]{fontenc}    % Výstupní kódování fontů
\usepackage[scaled]{helvet} % Načtení fontu Helvetica (klon Arialu)
\renewcommand{\familydefault}{\sfdefault} % Nastavení bezpatkového písma

% Barvy a grafika
\usepackage{xcolor}
\usepackage{graphicx}
\usepackage{float} 

% Tabulky
\usepackage{tabularx}
\usepackage{booktabs}
\usepackage{multirow}

% Čeština
\usepackage[czech]{babel}

% Balíčky pro formátování textu
\usepackage{parskip}       % Pro kontrolu mezer mezi odstavci
\usepackage{enumitem}
\setlist[itemize]{label=-, leftmargin=*} 

% Odkazy (musí být poslední)
\usepackage[hidelinks]{hyperref}
\hypersetup{
    pdftitle={Databázové systémy I - Projekt},
    pdfauthor={Andreas Brudovský},
    colorlinks=true,
    linkcolor=black,
    filecolor=black,
    urlcolor=black,
    citecolor=black
}

% Vypnutí mezer mezi odstavci a zapnutí odsazení prvního řádku
\setlength{\parindent}{0cm}
\setlength{\parskip}{1em}

% --- VLASTNÍ PŘÍKAZY ---
\newcommand{\uppersection}[1]{
    \subsection*{#1}
    \vspace{-0.5em}
}

% Definice sloupce pro tabulky (vycentrovaný X sloupec)
\newcolumntype{Y}{>{\centering\arraybackslash}X}

\begin{document}

% --- TITULNÍ STRANA ---
\begin{titlepage}
    \vspace*{5cm} 
    \centering
    {\LARGE Projekt do předmětu Databázové systémy I\par}
    \vspace{0.6cm}
    {\Huge\bfseries Databáze pro Rytmickou Hru \par}
    \vspace{0.6cm}
    {\Large (c) 2025, verze 1.0\par}
    \vspace{0.6cm}
    {\Large Andreas Brudovský, BRU0098\par}
    \vspace{0.6cm}
    {\Large VŠB -- Technická univerzita Ostrava\par}
    \vfill
\end{titlepage}

% --- OBSAH ---
\newpage
\tableofcontents
\newpage

% =========================================================================
% 1. POPIS SYSTÉMU
% =========================================================================
\section{Popis systému}

\subsection{Popis}
Tento projekt se zabývá návrhem databázové vrstvy pro počítačovou rytmickou hru. V této hře uživatelé do rytmu hudby mačkají klávesy, čímž získávají skóre. Mají na výběr z vícero "map" \newline reprezontovaných jako jednotlivé skladby, které mají dále variaci obtížností. Uživatelé mají možnost si oblíbené skladby / mapy ukládát do playlistů. Dále jsou odměňování ve formě úspěchů = achievementů.

\subsection{Rozsah systému}
Databáze pokrývá následující klíčové oblasti:
\begin{itemize}
    \item \textbf{Správa obsahu:} Evidence skladeb (Songs) a jejich herních variant (obtížností).
    \item \textbf{Hráčská data:} Uživatelé, jejich role a autentizační údaje.
    \item \textbf{Historie a statistiky:} Záznam každé odehrané hry (Session), včetně detailů jako přesnost a kombo.
    \item \textbf{Úspěchy:} Systém achievementů, které hráči získávají za splnění podmínek.
    \item \textbf{Uživatelský obsah:} Tvorba a správa vlastních seznamů skladeb.
\end{itemize}

\newpage

% =========================================================================
% 2. DATOVÉ MODELY
% =========================================================================
\section{Datové modely}

% 2.1 ERD (Zakomentováno dle tvého souboru)
%\subsection{Konceptuální datový model (ERD)}
%\begin{figure}[H]
%  \centering
%  \IfFileExists{erd_model.png}{
%    \includegraphics[width=1\textwidth]{erd_model.png}
%  }{
%      \framebox{\parbox{0.9\textwidth}{\centering
%        \vspace{4cm}
%        \textbf{ZDE VLOŽTE OBRÁZEK: erd\_model.png} \\
%        \small (Nahrajte soubor s tímto názvem do projektu v Overleafu)
%        \vspace{4cm}
%      }}
%  }
%  \caption{Konceptuální datový model}
%  \label{fig:erd}
%\end{figure}

% 2.2 Relační model
\subsection{Relační datový model}
\begin{figure}[H]
  \centering
  \IfFileExists{relacni_model.png}{
    \includegraphics[width=1\textwidth]{relacni_model.png}
  }{
      \framebox{\parbox{0.9\textwidth}{\centering
        \vspace{6cm}
        \textbf{ZDE VLOŽTE OBRÁZEK: relacni\_model.png} \\
        \small (Export z dbdiagram.io)
        \vspace{6cm}
      }}
  }
  \caption{Relační datový model}
  \label{fig:relational}
\end{figure}

\newpage

% 2.3 Popis významu tabulek
\subsection{Význam záznamů v tabulkách}

Níže je uveden popis toho, co reprezentuje jeden konkrétní záznam v každé navržené tabulce.

% --- ZDE JSME ODSTRANILI CHYBNÝ \begin{description} ---

\subsection{Tabulka Users}
\begin{table}[H]
\centering
\small
\begin{tabularx}{\textwidth}{|l|l|c|c|c|X|}
\hline
\textbf{Atribut} & \textbf{Dat. typ} & \textbf{Délka} & \textbf{Klíč} & \textbf{Null} & \textbf{Popis} \\ \hline
user\_id & INT & - & PK & Ne & ID uživatele \\ \hline
username & VARCHAR & 50 & UQ & Ne & Přihlašovací jméno \\ \hline
email & VARCHAR & 100 & UQ & Ne & Kontaktní email \\ \hline
password & VARCHAR & 255 & & Ne & Hash hesla \\ \hline
role & CHAR & 1 & & Ne & Role (P=Player, A=Admin) \\ \hline
\end{tabularx}
\end{table}

\subsection{Tabulka Songs}
\begin{table}[H]
\centering
\small
\begin{tabularx}{\textwidth}{|l|l|c|c|c|X|}
\hline
\textbf{Atribut} & \textbf{Dat. typ} & \textbf{Délka} & \textbf{Klíč} & \textbf{Null} & \textbf{Popis} \\ \hline
song\_id & INT & - & PK & Ne & ID skladby \\ \hline
title & VARCHAR & 100 & & Ne & Název skladby \\ \hline
artist & VARCHAR & 100 & & Ne & Interpret / Autor \\ \hline
bpm & INT & - & & Ano & Rychlost (Beats per min) \\ \hline
duration\_seconds & INT & - & & Ano & Délka v sekundách \\ \hline
file\_path & VARCHAR & 255 & & Ano & Cesta k souboru \\ \hline
\end{tabularx}
\end{table}

\subsection{Tabulka Song\_Variants}
\begin{table}[H]
\centering
\small
\begin{tabularx}{\textwidth}{|l|l|c|c|c|X|}
\hline
\textbf{Atribut} & \textbf{Dat. typ} & \textbf{Délka} & \textbf{Klíč} & \textbf{Null} & \textbf{Popis} \\ \hline
variant\_id & INT & - & PK & Ne & ID varianty \\ \hline
song\_id & INT & - & FK & Ne & ID Skladby \\ \hline
difficulty\_name & VARCHAR & 20 & & Ne & Název obtížnosti (Easy, Medium...) \\ \hline
note\_count & INT & - & & Ne & Počet not v levelu \\ \hline
max\_score & INT & - & & Ano & Maximální možné skóre \\ \hline
\end{tabularx}
\end{table}

\subsection{Tabulka Game\_Sessions}
\begin{table}[H]
\centering
\small
\begin{tabularx}{\textwidth}{|l|l|c|c|c|X|}
\hline
\textbf{Atribut} & \textbf{Dat. typ} & \textbf{Délka} & \textbf{Klíč} & \textbf{Null} & \textbf{Popis} \\ \hline
session\_id & INT & - & PK & Ne & ID Sessionu \\ \hline
user\_id & INT & - & FK & Ne & ID Uživatele - kdo hru hrál \\ \hline
variant\_id & INT & - & FK & Ne & Co hrál (obtížnost) \\ \hline
score & INT & - & & Ne & Dosažené skóre \\ \hline
accuracy & DEC & 5,2 & & Ano & Přesnost v \% \\ \hline
max\_combo & INT & - & & Ano & Nejvyšší kombo \\ \hline
is\_full\_combo & BIT & - & & Ano & Vše trefeno (0/1) \\ \hline
played\_at & DATETIME & - & & Ne & Čas odehrání (default na now()) \\ \hline
\end{tabularx}
\end{table}

\subsection{Tabulka Achievements}
\begin{table}[H]
\centering
\small
\begin{tabularx}{\textwidth}{|l|l|c|c|c|X|}
\hline
\textbf{Atribut} & \textbf{Dat. typ} & \textbf{Délka} & \textbf{Klíč} & \textbf{Null} & \textbf{Popis} \\ \hline
achievement\_id & INT & - & PK & Ne & ID achievementu \\ \hline
title & VARCHAR & 50 & & Ne & Název úspěchu \\ \hline
description & VARCHAR & MAX & & Ano & Popis podmínky \\ \hline
required\_points & INT & - & & Ano & Body nutné k získání \\ \hline
\end{tabularx}
\end{table}

\subsection{Tabulka User\_Achievements}
\begin{table}[H]
\centering
\small
\begin{tabularx}{\textwidth}{|l|l|c|c|c|X|}
\hline
\textbf{Atribut} & \textbf{Dat. typ} & \textbf{Délka} & \textbf{Klíč} & \textbf{Null} & \textbf{Popis} \\ \hline
id & INT & - & PK & Ne & Technické ID záznamu \\ \hline
user\_id & INT & - & FK & Ne & ID Uživatele \\ \hline
achievement\_id & INT & - & FK & Ne & ID Úspěchu \\ \hline
earned\_at & DATETIME & - & & Ano & Datum získání \\ \hline
\end{tabularx}
\end{table}

\subsection{Tabulka Playlists}
\begin{table}[H]
\centering
\small
\begin{tabularx}{\textwidth}{|l|l|c|c|c|X|}
\hline
\textbf{Atribut} & \textbf{Dat. typ} & \textbf{Délka} & \textbf{Klíč} & \textbf{Null} & \textbf{Popis} \\ \hline
playlist\_id & INT & - & PK & Ne & ID playlistu \\ \hline
user\_id & INT & - & FK & Ne & Vlastník playlistu \\ \hline
name & VARCHAR & 50 & & Ne & Název playlistu \\ \hline
is\_public & BIT & - & & Ano & Veřejný (0/1) \\ \hline
\end{tabularx}
\end{table}

\subsection{Tabulka Playlist\_Items}
\begin{table}[H]
\centering
\small
\begin{tabularx}{\textwidth}{|l|l|c|c|c|X|}
\hline
\textbf{Atribut} & \textbf{Dat. typ} & \textbf{Délka} & \textbf{Klíč} & \textbf{Null} & \textbf{Popis} \\ \hline
playlist\_id & INT & - & FK & Ne & ID playlistu \\ \hline
song\_id & INT & - & FK & Ne & ID Skladby \\ \hline
\end{tabularx}
\end{table}

\end{document}